\input{formato.tex}
\begin{document}
\def\title{Competitive Programing Reference}
.\\[0.2cm]
\centering{\LARGE\textbf{JakeMate14}} \\[0.5cm]
\tableofcontents\newpage

\section{Busqueda Binaria}
	\subsection{LB y UP}
		\cppfile{BusquedaBinaria/BinarySearch.cpp}
	\subsection{Busqueda con numeros reales}
		\cppfile{BusquedaBinaria/Reales.cpp}

\section{Estructuras de Datos}
	\subsection{Segment Tree}
		\cppfile{EstructurasDeDatos/Segment_Tree.cpp}
	\subsection{Segment Tree Lazy}
		\cppfile{EstructurasDeDatos/Segment_Tree_Lazy.cpp}
	\subsection{Suffix Array}
		\cppfile{EstructurasDeDatos/suffixArray.cpp}

\section{Grafos}
	\subsection{DFS en Grids}
		\cppfile{Grafos/dfsGrids.cpp}

\section{Arboles}
	\subsection{Distancia de cada nodo desde la raiz}
		\cppfile{Arboles/distDesdeRaiz.cpp}
	\subsection{Distancia de la raiz a cualquier nodo}
		\cppfile{Arboles/distMaximaCadaNodo.cpp}
	\subsection{Diametro de un arbol}
		\cppfile{Arboles/diametro.cpp}
	\subsection{Tamaño del subarbol del nodo x}
		\cppfile{Arboles/tamSubArbol.cpp}


\section{Math}

\section{DP}
	\subsection{Coin Problem}
		\cppfile{DP/coins.cpp}
	\subsection{Digitos}
		\cppfile{DP/Digitos/template.cpp}

\section{Geometry}


\section{Strings}


\section{Flow}


\section{Other}
	\subsection{Template}
		\cppfile{Template/Template.cpp}


\end{document}

